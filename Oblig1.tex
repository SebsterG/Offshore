\documentclass[DIV=calc, paper=a4, fontsize=11pt, twocolumn]{scrartcl}	 % A4 paper and 11pt font size

\usepackage{lipsum} % Used for inserting dummy 'Lorem ipsum' text into the template
\usepackage[english]{babel} % English language/hyphenation
\usepackage[protrusion=true,expansion=true]{microtype} % Better typography
\usepackage{amsmath,amsfonts,amsthm} % Math packages
\usepackage[svgnames]{xcolor} % Enabling colors by their 'svgnames'
\usepackage[hang, small,labelfont=bf,up,textfont=it,up]{caption} % Custom captions under/above floats in tables or figures
\usepackage{booktabs} % Horizontal rules in tables
\usepackage{fix-cm}	 % Custom font sizes - used for the initial letter in the document

\usepackage{sectsty} % Enables custom section titles
\allsectionsfont{\usefont{OT1}{phv}{b}{n}} % Change the font of all section commands

\usepackage{fancyhdr} % Needed to define custom headers/footers
\pagestyle{fancy} % Enables the custom headers/footers
\usepackage{lastpage} % Used to determine the number of pages in the document (for "Page X of Total")

% Headers - all currently empty
\lhead{}
\chead{}
\rhead{}

% Footers
\lfoot{}
\cfoot{}
\rfoot{\footnotesize Page \thepage\ of \pageref{LastPage}} % "Page 1 of 2"

\renewcommand{\headrulewidth}{0.0pt} % No header rule
\renewcommand{\footrulewidth}{0.4pt} % Thin footer rule

\usepackage{lettrine} % Package to accentuate the first letter of the text
\newcommand{\initial}[1]{ % Defines the command and style for the first letter
\lettrine[lines=3,lhang=0.3,nindent=0em]{
\color{DarkGoldenrod}
{\textsf{#1}}}{}}

%----------------------------------------------------------------------------------------
%	TITLE SECTION
%----------------------------------------------------------------------------------------

\usepackage{titling} % Allows custom title configuration

\newcommand{\HorRule}{\color{DarkGoldenrod} \rule{\linewidth}{1pt}} % Defines the gold horizontal rule around the title

\pretitle{\vspace{-30pt} \begin{flushleft} \HorRule \fontsize{50}{50} \usefont{OT1}{phv}{b}{n} \color{DarkRed} \selectfont} % Horizontal rule before the title

\title{Offshore Technology  Assignment 1} % Your article title

\posttitle{\par\end{flushleft}\vskip 0.5em} % Whitespace under the title

\preauthor{\begin{flushleft}\large \lineskip 0.5em \usefont{OT1}{phv}{b}{sl} \color{DarkRed}} % Author font configuration

\author{Sebastian Gjertsen, } % Your name

\postauthor{\footnotesize \usefont{OT1}{phv}{m}{sl} \color{Black} % Configuration for the institution name
University of Oslo % Your institution

\par\end{flushleft}\HorRule} % Horizontal rule after the title

\date{} % Add a date here if you would like one to appear underneath the title block

%----------------------------------------------------------------------------------------

\begin{document}

\maketitle % Print the title

\thispagestyle{fancy} % Enabling the custom headers/footers for the first page 

%----------------------------------------------------------------------------------------
%	ABSTRACT
%----------------------------------------------------------------------------------------

% The first character should be within \initial{}


%----------------------------------------------------------------------------------------
%	ARTICLE CONTENTS
%----------------------------------------------------------------------------------------





%------------------------------------------------

\section*{Petrophysical logs}
To find hydrocarbons buried underneath the sea bed we can drill and study cores from drilling or we can use petrophysical logging. The latter is not only cheaper but gives information that cannot be found by drilling. The different logs are used by combining the results to deduce wether or not hydrocarbons are present .These logs are categorized into:
\subsection*{Selfpotential}
This log utilizes differences in electric potential. Giving information on shale boundaries which can give us information on the reservoir rock. It also gives us information on $\rho_w$ which helps us calculate the pore space occupied by the hydrocarbons. The SP can give a constant value when the sand is clean and bed is thick. We then call it a static SP. This is used to determine the resistivity  $\rho_w$, which we will talk about in the next section.
\subsection*{Resistivity}
The resistivity measures resistance of an electrical current through a rock. Which can help us measure the pore space that is occupied by the water in the rock, which we call the water saturation. This value is used to find the saturation of hydrocarbons in a given rock formation. A rock saturated with hydrocarbons will typically have a high SP and high resistivity.
\subsection*{Radioactivity}
As the name indicates this log uses radioactivity in the rock to determine the rock type. Where shale rocks have a high radioactivity, and coal has low radioactivity. This log can also be used to monitor the cement, by injecting radioactive material into the cement. The rock is identified through the radioactivity log by looking at the density and porosity of a given rock formation. The radioactivity reader is usually held against the side of the dug hole with two detectors so to get an accurate reading.
\subsection*{Neutron}
This logging type is similar to that of radioactive logging, only that it utilizes the effects of neutrons. The neutrons are sent into the rock and clashes with atoms causing $\gamma$ rays to be sent back and read. The stronger the signal the more hydrogen is present. Which is an integral part in both water and hydrocarbons. This log is therefore used to measure the porosity in which the hydrocarbons reside.
\subsection*{Sonic}
This logs the variation of acoustic sound waves sent down and reflected, which is also known as seismic. The different times at which these waves return determines the porosity, which again can tell us something about the rock.
\section*{Top reservoir}
As we see from the log the top reservoir is at about 2745 m. We know this by reading the gamma ray and seeing a change around 2745 m. This is also confirmed by the density and porosity decreasing in that area. Lastly we see that the electrical resistance is higher also in that area. 
\section*{Hydrocarbons}
From the logs we can see that in the top and bottom we have shale rock. We see this from the density and neutron porosity both dropping off at around 2745 m. The electrical resistance lets us know that we have hydrocarbons in the previous mentioned area. From reading the Neutron porosity log, I believe that we have oil in our reservoir. If it was gas then the yellow area so to speak would be bigger. 
\section*{Water and Oil in our well}
The point where the oil is in contact with water is at about 2773m.  And we see that we have a transition zone from 2773m to 2786m. At the top of the transition zone we have mostly oil and a water saturation close to zero or a constant value. Since we can have 100 \% water saturation but never 100 \% oil saturation, cause there will always be some water in the pores. At the bottom of the transition zone is the top of the free water level.
The reason we have a transition interval is the changing of rock structure which allows for more or less water in the rock. Another effect is also the capillary pressure in the pores, causing there to be a drop off in water saturation. The transition zones thickness will be decided by the matrix of the rock. If the rock is more complex, than there will be a mixture of fluids in different pore types

\section*{Free water level and oil water contact }
Since water, oil and gas have different density they will tend to layer on top of each other. With gas being the lightest and water being the heaviest.
The free water level is the highest level in which the pressure of the hydrocarbons is the same as the water. Or where the water saturation is 100\%. The oil water is the point where water and oil meet. This is where the transition zone starts. The boundaries can be determined by studying pressure gradients. Its obviously important to discover where the oil starts and where oil and water contact. Since we want to produce mainly hydrocarbons and not water.

\section*{Net/Gross}
The Gross hydrocarbons portion of this well is from 2745 to 2774. So its about 30 meters. From my estimation the net portion is about 18 meters. So the net/gross is about 18/30.

\section*{Types of rocks}
In the top and bottom of the log we have shale, where the green area is, in the neutron and bulk density log. At 2754 meters it looks like we have a calcite rock. And the same at 2777 meters. We see this first in the sonic log, where there are two big spikes. Letting us know that there is a change in material. We can also see that the density goes up and neutron porosity down which is telling us that its a calcite. The resistance also spikes which again tells us that there is a different rock there. The oil is in silty/shaly sand. This is read of the neutron and bulk density, and compared to the scale found in lecture slides. And also when comparing gamma ray log.

\section*{Pore-volume}
The pore volume states the size of the volume that the hydrocarbons take inside the rock. Because of this we want a porous that can transport hydrocarbons. But that also can act as a seal for the oil and gas. The pore volume is dependent on effective porosity, which is porosity that contributes to effective fluid flow and on the the thickness of the rock layers. This is obviously dependent on what type of rock or sand is in the reservoir .Since the pores are always filled with fluids the pressure of the fluids will influence the pores.


\section*{Resource and Reserves}
The resource of a field is all estimated quantities of hydrocarbons under the surface as well as produced hydrocarbons. The reserves are the proven actual amount of hydrocarbons that can be produced within a certain field. So the reserve is within the resource. The reserves can expand as economical or other factors change. Such like when a company abandons a oil field because its not longer economically viable for them, it may change for another company at another time.
\section*{STOOIP and GIIP}
STOOIP stand for standard tank original oil in place. This acronym stands for all the oil in a reservoir before production starts. GIIP stands for Gas Initially In Place, which is the total volume of gas before production.

\section*{Recovery factors}
The factors that govern how much oil we can extract is mainly dependent on the drive mechanism of the reservoir. These mechanism are


\begin{itemize}
\item Gas expansion - Free gas in the reservoir expands and pushes the oil to production
\item Water driver - Water from aquifer pushes the oil to the borehole as the pressure around the borehole drops
\item Solution gas - Gas trapped inside the oil releases and pushes the oil to production
\item Rock drive - Decline in fluid pressure, increases pressure on the solids, which drives the fluid out.
\end{itemize}
All of these factor contribute how much oil we can recover.
For instance in some reservoirs we do not have a good water driver, which is the best driver. We have to then rely on gas expansion to drive the oil out. This will lead to a lower recovery than if the oil is driven by water. 
\newline
If we solely rely on natural drivers to produce oil, the recovery factor would not be very high. To help with this, pumping water or gas down next to the well, to maintain pressure, will increase the production. 
\newline
The porosity of the rock will obviously also make a difference, since this allows the hydrocarbons to move through the rock. The oil could combined with this be to viscous ,or thick. To tackle this we can heat, or steam, the oil to make it easier to extract.


\section*{Porosity and Saturation}
In table 1 we can see the porosity and oil saturation calculated at different heights. This was done by using the formula for porosity: 
$$ \frac{\rho_{ma}-\rho_{b}}{\rho_{ma}-\rho_f}     $$ where $\rho_{ma}$ is the density of the rock matrix, $\rho_{b}$ is the density read of the log, and  $\rho_{f}$ is the fluid density. 
\newline
The porosity is how much empty space there is in a rock for fluids to reside. These calculations were done by looking at the different values on the log and using some predetermined values. 
\newline
The oil saturation was done by using the formula for water saturation :
$$ S_w = \frac{1}{\Phi}\sqrt{\frac{R_w}{R_t}}  $$
To find the oil saturation we had to: $S_h = 1-S_w$
I also calculated the water resistivity to be 0.36, and used this number for my calculations.
The calculated saturation for the depths 2750m and 2763m seems realistic. While the saturation of $0.74$ at $2782$m seems too much since there is a transition zone of oil and water. I would think that this number is much closer to zero. And the last saturation is zero since there is only water. 
\newline
The accuracy of these calculations are questionable. First of all the values I used are just the ones i read of the log. I do not have the exact scalar value measurements at every point. There might also be instrument measurement uncertainty. 
To get the most accurate porosity value. We would have to get a core sample from the rock, and physically measure. Though we might damage the pores in the process of getting the sample. 
We would also want a sample of the rock to find out the oil saturation.	





\begin{table}
\caption{Porosity and Oil Saturation}
\centering
\begin{tabular}{llr}
\toprule
Meters & Porosity & Oil Saturation \\
\midrule
$2750$ & $0.23$ & $0.79$ \\
$2763$ & $0.37$ & $0.88$ \\
$2782$ & $0.26$ & $0.74$ \\
$2790$ & $0.125$ & $0$ \\

\bottomrule
\end{tabular}
\end{table}




%----------------------------------------------------------------------------------------
%	REFERENCE LIST
%----------------------------------------------------------------------------------------

\begin{thebibliography}{99} % Bibliography - this is intentionally simple in this template

\bibitem[Cambridge, 2000]{}
Musset, Khan (2000).
\newblock Looking into the earth
\newblock {\em Subsurface Geophysics }, 18:285--305.
 
\end{thebibliography}

%----------------------------------------------------------------------------------------

\end{document}