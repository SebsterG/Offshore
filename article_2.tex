\documentclass[twoside]{article}

\usepackage{lipsum} % Package to generate dummy text throughout this template

\usepackage[sc]{mathpazo} % Use the Palatino font
\usepackage[T1]{fontenc} % Use 8-bit encoding that has 256 glyphs
\linespread{1.05} % Line spacing - Palatino needs more space between lines
\usepackage{microtype} % Slightly tweak font spacing for aesthetics

\usepackage[hmarginratio=1:1,top=32mm,columnsep=20pt]{geometry} % Document margins
\usepackage{multicol} % Used for the two-column layout of the document
\usepackage[hang, small,labelfont=bf,up,textfont=it,up]{caption} % Custom captions under/above floats in tables or figures
\usepackage{booktabs} % Horizontal rules in tables
\usepackage{float} % Required for tables and figures in the multi-column environment - they need to be placed in specific locations with the [H] (e.g. \begin{table}[H])
\usepackage{hyperref} % For hyperlinks in the PDF

\usepackage{lettrine} % The lettrine is the first enlarged letter at the beginning of the text
\usepackage{paralist} % Used for the compactitem environment which makes bullet points with less space between them

\usepackage{abstract} % Allows abstract customization
\renewcommand{\abstractnamefont}{\normalfont\bfseries} % Set the "Abstract" text to bold
\renewcommand{\abstracttextfont}{\normalfont\small\itshape} % Set the abstract itself to small italic text

\usepackage{titlesec} % Allows customization of titles
\renewcommand\thesection{\Roman{section}} % Roman numerals for the sections
\renewcommand\thesubsection{\Roman{subsection}} % Roman numerals for subsections
\titleformat{\section}[block]{\large\scshape\centering}{\thesection.}{1em}{} % Change the look of the section titles
\titleformat{\subsection}[block]{\large}{\thesubsection.}{1em}{} % Change the look of the section titles

\usepackage{fancyhdr} % Headers and footers
\pagestyle{fancy} % All pages have headers and footers
\fancyhead{} % Blank out the default header
\fancyfoot{} % Blank out the default footer
\fancyhead[C]{} % Custom header text
\fancyfoot[RO,LE]{\thepage} % Custom footer text

%----------------------------------------------------------------------------------------
%	TITLE SECTION
%----------------------------------------------------------------------------------------

\title{\vspace{-15mm}\fontsize{24pt}{10pt}\selectfont\textbf{Abels Lifting Operations}} % Article title

\author{
\large
\textsc{Sebastian Gjertsen} \\[2mm] % Your name
\normalsize University of Oslo \\ % Your institution
\normalsize \href{mailto:sebastgj@math.uio.no}{sebastgj@math.uio.no} % Your email address
\vspace{-5mm}
}
\date{}

%----------------------------------------------------------------------------------------

\begin{document}

\maketitle % Insert title

\thispagestyle{fancy} % All pages have headers and footers

%----------------------------------------------------------------------------------------
%	ABSTRACT
%----------------------------------------------------------------------------------------

\begin{abstract}

\noindent 
Lifting operations of 16 torpedo anchors. Safely and efficently from onshore to seabed

\end{abstract}

%----------------------------------------------------------------------------------------
%	ARTICLE CONTENTS
%----------------------------------------------------------------------------------------

\begin{multicols}{2} % Two-column layout throughout the main article text

\subsection{Introduction*}

\lettrine[nindent=0em,lines=3]{A}bels lifting operations is a long standing operator in the lifting and installing offshore industry. We have many succesfull offshore installations over the years. We offer installations in a secure and effient manner. This appendix will cover the steps involved in installing 16 torpedo anchors. Starting from picking up the anchors onshore to installation on seafloor.


%------------------------------------------------

\subsection{Onshore and Load out}
In this operation we need only a crane vessel, which will also serve as a transportation vessel. By only using one vessel we will save money and time. But it will need to be carefully planned in regards to deckspace and lift out.
The first part of the operation will be to practice the lift out from the vessel into the water. As this is a crucial part of the operation where many errors can occur. When we lift an anchor weighing 50 tons from the floor of a vessel and horizontally out to sea, the shift in the vessels balance may seriously be compromised. This fact and the dynamic forces of the waves and weather, will make this a delicate procedure. I therefore strongly recommend that a short amount of time is used on a practice lift while the vessel is keyside. This will help us in knowing what to expect when we get out to sea. This could in worst case prevent loss of lives and or precious equipment.

\subsection{Transportation}
The vessel we are using for transportation has been carefully calculated to withstand the force of 800 tons of steel. The vessel will also be searched for cracks















\begin{compactitem}
\item Donec dolor arcu, rutrum id molestie in, viverra sed diam
\item Curabitur feugiat
\item turpis sed auctor facilisis
\item arcu eros accumsan lorem, at posuere mi diam sit amet tortor
\item Fusce fermentum, mi sit amet euismod rutrum
\item sem lorem molestie diam, iaculis aliquet sapien tortor non nisi
\item Pellentesque bibendum pretium aliquet
\end{compactitem}


%------------------------------------------------

\section{Results}

\begin{table}[H]
\caption{Example table}
\centering
\begin{tabular}{llr}
\toprule
\multicolumn{2}{c}{Name} \\
\cmidrule(r){1-2}
First name & Last Name & Grade \\
\midrule
John & Doe & $7.5$ \\
Richard & Miles & $2$ \\
\bottomrule
\end{tabular}
\end{table}

\lipsum[5] % Dummy text

\begin{equation}
\label{eq:emc}
e = mc^2
\end{equation}

\lipsum[6] % Dummy text

%------------------------------------------------

\section{Discussion}

\subsection{Subsection One}

\lipsum[7] % Dummy text

\subsection{Subsection Two}

\lipsum[8] % Dummy text

%----------------------------------------------------------------------------------------
%	REFERENCE LIST
%----------------------------------------------------------------------------------------

\begin{thebibliography}{99} % Bibliography - this is intentionally simple in this template

\bibitem[Figueredo and Wolf, 2009]{Figueredo:2009dg}
Figueredo, A.~J. and Wolf, P. S.~A. (2009).
\newblock Assortative pairing and life history strategy - a cross-cultural
  study.
\newblock {\em Human Nature}, 20:317--330.
 
\end{thebibliography}

%----------------------------------------------------------------------------------------

\end{multicols}

\end{document}
